\chapter*{Article 2}
\article{Self-testing two-qubit maximally entangled states from generalized Clauser-Horne-Shimony-Holt tests}

\centering
\textrm{\LARGE Self-testing two-qubit maximally entangled states from generalized Clauser-Horne-Shimony-Holt tests}

\vspace{2cm}

\normalsize
Xavier Valcarce$^1,2$, Julian Zivy$^1,2$, Nicolas Sangouard$^1,2$, and Pavel Sekatski$^2$
\bigbreak

{\footnotesize
	$^1$ Université Paris-Saclay, CEA, CNRS, Institut de Physique Théorique, 91191 Gif-sur-Yvette, France \\
	$^2$ Departement Physik, Universität Basel, Klingelbergstraße 82, 4056 Basel, Schweiz
}

\raggedright
\bigbreak
\faLink \quad \href{https://journals.aps.org/prresearch/abstract/10.1103/PhysRevResearch.4.013049}{Phys. Rev. Research 4, 013049} \\
\faLink \quad \href{https://arxiv.org/abs/2011.03047}{arXiv preprint: 2011.03047}
\vspace{1cm}

\centering
\textbf{Abstract}
\bigbreak

Device-independent certification, also known as self-testing, aims at guaranteeing the proper functioning of untrusted and uncharacterized devices.
For example, the quality of an unknown source expected to produce two-qubit maximally entangled states can be evaluated in a bi-partite scenario, each party using two binary measurements.
The most robust approach consists in deducing the fidelity of produced states with respect to a two-qubit maximally entangled state from the violation of the CHSH inequality.
In this paper, we show how the self-testing of two-qubit maximally entangled states is improved by a refined analysis of measurement statistics.
The use of suitably chosen Bell tests, depending on the observed correlations, allows one to conclude higher fidelities than ones previously known.
In particular, nontrivial self-testing statements can be obtained from correlations that cannot be exploited by a CHSH-based self-testing strategy.
Our results not only provide novel insight into the set of quantum correlations suited for self-testing, but also facilitate the experimental implementations of device-independent certifications.
