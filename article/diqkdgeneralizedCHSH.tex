\chapter*{Article 3}
\article{Device-independent quantum key distribution from generalized CHSH inequalities}

\begin{center}
\textrm{\LARGE Device-independent quantum key distribution from generalized CHSH inequalities}

\vspace{2cm}

\normalsize
Pavel Sekatski$^{1}$, Jean-Daniel Bancal$^{2}$, Xavier Valcarce$^{3}$, Ernest Y.-Z. Tan$^{4}$, Renato Renner$^{4}$, and Nicolas Sangouard$^{1,3}$
\bigbreak

{\footnotesize
	$^1$ Departement Physik, Universität Basel, Klingelbergstraße 82, 4056 Basel, Schweiz \\
	$^2$ Department of Applied Physics, University of Geneva, Chemin de Pinchat 22, 1211 Geneva, Switzerland \\
	$^3$ Université Paris-Saclay, CEA, CNRS, Institut de Physique Théorique, 91191 Gif-sur-Yvette, France \\
	$^4$ Institute for Theoretical Physics, ETH Zürich, 8093 Zürich, Switzerland
}

\raggedright
\bigbreak
\faLink \quad \href{https://quantum-journal.org/papers/q-2021-04-26-444/}{Quantum, volume 5, page 444} \\
\faLink \quad \href{https://arxiv.org/abs/2009.01784v4}{arXiv preprint: 2009.01784v4}
\vspace{1cm}

\centering
\textbf{Abstract}
\bigbreak

Device-independent quantum key distribution aims at providing security guarantees even when using largely uncharacterised devices.
In the simplest scenario, these guarantees are derived from the CHSH score, which is a simple linear combination of four correlation functions.
We here derive a security proof from a generalisation of the CHSH score, which effectively takes into account the individual values of two correlation functions.
We show that this additional information, which is anyway available in practice, allows one to get higher key rates than with the CHSH score.
We discuss the potential advantage of this technique for realistic photonic implementations of device-independent quantum key distribution.

\end{center}
