\chapter*{Article 4}
\article{Automated design of quantum optical experiments for device-independent quantum key distribution}

\begin{center}
\textrm{\LARGE Automated design of quantum optical experiments for device-independent quantum key distribution}

\vspace{2cm}

\normalsize
Xavier Valcarce$^{1}$, Pavel Sekatski$^{2}$, Elie Gouzien$^{1}$, Alexey Melnikov$^{3}$, Nicolas Sangouard$^{1}$
\bigbreak

{\footnotesize
	$^1$ Université Paris-Saclay, CEA, CNRS, Institut de Physique Théorique, 91191 Gif-sur-Yvette, France \\
	$^2$ Department of Applied Physics, University of Geneva, Chemin de Pinchat 22, 1211 Geneva, Switzerland \\
	$^3$ Terra Quantum AG, 9000 St Gallen, Switzerland
}

\raggedright
\bigbreak
\faLink \quad \href{https://arxiv.org/abs/2209.06468}{arXiv preprint: 2209.06468}
\vspace{1cm}

\centering
\textbf{Abstract}
\bigbreak

Device-independent quantum key distribution (DIQKD) reduces the vulnerability to side-channel attacks of standard QKD protocols by removing the need for characterized quantum devices.
The higher security guarantees come however, at the price of a challenging implementation.
Here, we tackle the question of the conception of an experiment for implementing DIQKD with photonic devices.
We introduce a technique combining reinforcement learning, optimisation algorithm and a custom efficient simulation of quantum optics experiments to automate the design of photonic setups maximizing a given function of the measurement statistics.
Applying the algorithm to DIQKD, we get unexpected experimental configurations leading to high key rates and to a high resistance to loss and noise.
These configurations might be helpful to facilitate a first implementation of DIQKD with photonic devices and for future developments targeting improved performances.

\end{center}
