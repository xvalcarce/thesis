\chapter*{Article 1}
\article{What is the minimum CHSH score certifying that a state resembles the singlet?}

\centering
\textrm{\LARGE What is the minimum CHSH score certifying that a state resembles the singlet?}

\vspace{2cm}

\normalsize
Xavier Valcarce$^1$, Pavel Sekatski$^1$, Davide Orsucci$^1$, Enky Oudot$^{1,2}$, Jean-Daniel Bancal$^{1,2}$, and Nicolas Sangouard$^1$
\bigbreak

{\footnotesize
$^1$ Departement Physik, Universität Basel, Klingelbergstraße 82, 4056 Basel, Schweiz \\
$^2$ Département de Physique Appliquée, Université de Genève, 1211 Genève, Suisse
}

\raggedright
\bigbreak
\faLink \quad \href{https://quantum-journal.org/papers/q-2020-03-23-246/}{Quantum, volume 4, page 246} \\
\faLink \quad \href{https://arxiv.org/abs/1910.04606}{arXiv preprint: 1910.04606}
\vspace{1cm}

\centering
\textbf{Abstract}
\bigbreak

A quantum state can be characterized from the violation of a Bell inequality.
The well-known CHSH inequality for example can be used to quantify the fidelity (up to local isometries) of the measured state with respect to the singlet state.
In this work, we look for the minimum CHSH violation leading to a non-trivial fidelity.
In particular, we provide a new analytical approach to explore this problem in a device-independent framework, where the fidelity bound holds without assumption about the internal working of devices used in the CHSH test.
We give an example which pushes the minimum CHSH threshold from $\approx 2.0014$ to $\approx 2.05$, far from the local bound.
This is in sharp contrast with the device-dependent (two-qubit) case, where entanglement is one-to-one related to a non-trivial singlet fidelity.
We discuss this result in a broad context including device-dependent/independent state characterizations with various classical resources.
