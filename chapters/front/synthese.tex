\chapter{Synthèse en français}

\section{Motivations et context}

La physique a connu un changement radical de paradigme au début du XXe siècle.
La perspective d'un monde continu décrit à la fois par les équations différentielles de la mécanique Newtonienne et par les équations de Maxwell-Boltzamm, fut confronté à un monde discret, dépeint par la quantification de l'énergie postulée par Max Planck.
De plus, notre compréhension intuitive des objets physiques, caractérisés de manière locale et déterministe dans l'espace et le temps, a été balayée par la nature fondamentalement probabiliste de la théorie quantique.
Cela engendra de nombreux débats philosophiques sur l'interprétation de la physique quantique, dont certains sont encore d'actualité aujourd'hui.
Par ailleurs, de nouveaux modèles furent construits à partir de la physique quantique, nous permettant d'étendre notre compréhension du monde.
De la chimie quantique au modèle standard de la physique des particules, ces modèles ont connu un succès éclatant pour la prédiction des phénomènes naturels de l'Univers.
En parallèle, une myriade de nouvelles technologies issues la mécanique quantique virent le jour.
Le laser, les transistors ou encore les IRMs sont des exemples connus parmi ces applications.
Ces nouveaux modèles et technologies ont marqué la \textit{première révolution quantique}.

\medbreak

Dans les années 80, une nouvelle direction fut prise : nous ne cherchons plus seulement à comprendre les objets quantiques, mais à les construire et à les contrôler.
Ce changement fut provoqué par la réalisation que certaines propriétés quantiques peuvent être utilisée pour augmenter nos capacités de calcul et de communication.
En exploitant l'intrication quantique, la communication quantique promet des preuves de cryptographie basée sur les lois fondamentales de la physique, tandis que l'informatique quantique suggère des gains exponentiels de performance pour la réalisation de certaines tâches.
De plus, les capteurs quantiques permettent d'atteindre une précision jamais atteinte pour la résolution de certaines grandeurs physiques, et les simulateurs quantiques nous aideront à résoudre des problèmes au-delà de nos capacités de calcul actuelles.
Dans ce contexte, des organisations nationales et internationales réunissent les ressources nécessaires au développement et à l'évolution de ces technologies quantiques.
À long terme, nous pourrions voir apparaître un internet quantique reliant ordinateurs, capteurs et simulateurs quantiques, le tout communicant de manière sécurisée grâce à la communication quantique.

\medbreak

Pour la réalisation d'un réseau complexe de technologies quantiques, la capacité de certifier ces technologies quantiques apparaît comme primordiale. 
La certification de ressources quantiques devrait permettre la détection rapide d'erreurs et attester du comportement des appareils quantiques, ce qui est critique pour les tâches sensibles en matière de sûreté.
Certains protocoles de certification peuvent être réalisés dans une approche \guillemotleft boîte noire \guillemotright, c'est-à-dire sans hypothèse sur le fonctionnement interne des appareils quantiques, considérés comme des boîtes noires. 
L'exemple le plus notable est le \textit{self-testing}, un protocole permettant de garantir la présence d'états quantiques et de mesures spécifiques.
Pour certaines applications quantiques, il est également possible de concevoir des protocoles « boîte noire » dont leur succès est conditionné à la présence de ressources particulières.
La distribution quantique de clé de type \guillemotleft boîte noire \guillemotright est un exemple caractéristique de ce type de protocoles. Elle permet à deux parties de partager une clé secrète de chiffrement dont la preuve de sécurité ne dépend pas des appareils utilisés. 
Une attaque par une partie adverse ou des défauts dans les appareils utilisés résultent en des interférences détectables par le protocole, qui s'interrompra.

\medbreak

Dans cette thèse, nous étudions ces protocoles \guillemotleft boîte noire \guillemotright, avec une attention particulière portée sur la facilitation de leur réalisation expérimentale.
Bien que ces protocoles soient prometteurs, leur mise en œuvre nécessite le respect de nombreuses conditions. 
Nous commençons par déterminer les ressources fondamentales nécessaires à leur réalisation, puis nous améliorons ces protocoles pour les rendre plus résistants aux pertes, ce qui réduit les exigences sur leur réalisation expérimentale. 
Enfin, nous proposons une méthode pour automatiser la conception d'expériences d'optique quantique. 
Cela nous permettra d'obtenir de nouveaux designs bien adaptés à la réalisation de ces protocoles.


\section{Résumé des contributions}

\paragraph{Certification «boîte noire» de ressources quantiques} 

Le self-testing certifie la présence de certaines ressources quantiques à partir de statistiques de mesures.
Nous examinons le \textit{robust self-testing} du singlet, le protocole permettant, malgré la présence de bruit et de pertes, la certification d'un état de deux qubits maximallement intriqué.

\medbreak

\textbf{Article 1~\cite{Valcarce2020}:} 
Nous quantifions les ressources nécessaires pour appliquer des protocoles de self-testing en présence d'imperfections.
Nous nous concentrons sur le self-testing basé sur une seule condition que doivent respecter les statistiques de mesures ; la violation de l'inégalité CHSH.
Ce travail conduit à la découverte d'une condition nécessaire : une violation minimale en dessous de laquelle le robust self-testing échoue.
Cette limite fondamentale améliore notre compréhension du self-testing, définie des exigences expérimentales claires et suggère des orientations pour de futurs protocoles.

\textbf{Article 2~\cite{Valcarce2022}:} 
À partir d'une analyse affinée des statistiques de mesure, nous dérivons de nouveaux self-tests robustes pour le singlet.
Le nouveau protocole que nous proposons offre une meilleure robustesse aux pertes et facilite ainsi la réalisation expérimentale du self-testing du singlet.

\paragraph{Distribution quantique de clé de type \guillemotleft boîte noire \guillemotright}

La distribution quantique de clé de type \guillemotleft boîte noire \guillemotright permet à deux parties de partager une clé de manière sûre.
La sécurité de ce protocole repose sur des conditions que doivent satisfaire les statistiques de mesures.

\medbreak

\textbf{Article 3~\cite{Sekatski2021}:}
Nous formulons une nouvelle preuve de sécurité qui limite plus justement l'information qu'un adversaire peut obtenir sur la clé.
Notre preuve est basée sur deux fonctions de corrélation au lieu d'une seule combinaison linéaire de toutes les statistiques.
Le taux de clé qui en découle est plus élevé pour les implémentations réalistes basées sur l'optique quantique et pour les états partiellement intriqués. 
Cependant notre approche ne fournit pas une meilleure efficacité critique.

\textbf{Article 4~\cite{Valcarce2022b}:} 
En combinant l'apprentissage machine et un framework de simulation de circuits photoniques~\cite{Valcarce2021}, nous automatisons la conception d'expériences photoniques.
Appliquée à la distribution quantique de clé de type \guillemotleft boîte noire \guillemotright, notre méthode résulte en de nouveaux designs d'expériences réalistes, offrant à la fois un taux de clé plus élevé et une tolérance plus élevée aux pertes par rapport aux propositions photoniques précédemment connues.

