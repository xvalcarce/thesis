\chapter{Thesis overview}

\section{Motivations and background}

Physics went through a drastic paradigm change at the beginning of the 20th century.
The perspective of a continuous world, provided both by differential equations in Newtonian mechanics and by Maxwell-Boltzmann equations, clashed with the discrete world depicted by the quantization of energy postulated by Max Planck.
Moreover, our intuitive comprehension of physical objects, locally and deterministically characterized in space and time, was blown away by the fundamentally probabilistic nature of quantum theory.
This led to numerous philosophical debates on the interpretation of the quantum world, with some of them still thriving nowadays.
Additionally, new models widening the understanding of our surrounding were constructed from quantum physics.
From quantum chemistry to the Standard model, all of these models met unprecedented success in predicting natural phenomena.
Furthermore, these models enabled a myriad of new technological applications.
The laser, transistors or MRIs are some famous examples of such applications.
Together, these models and technologies constitute what is known today as the \textit{first quantum revolution}.

\medbreak

The 1980s marked a shift in focus from understanding quantum objects to engineering and controlling them.
This change was driven by the realization that some quantum behaviors could be harnessed to enhance our information processing capabilities, and led to the emergence of a new field, \textit{quantum information processing}.
Leveraging quantum entanglement, quantum communication promises cryptographic proofs based on the fundamental law of physics, while quantum computing suggests exponential performance gains for some tasks.
Moreover, quantum sensing enables the resolution of physical properties with an unprecedented precision and quantum simulators will help us solve open questions that are beyond reach to our current computing capacity. 
In this scope, national and international alliances are pooling resources to develop and scale up quantum technologies.
On the long-term, we could see the apparition of a quantum internet, connecting quantum computers, simulators and sensors, and secured by quantum communication.

\medbreak

In the quest towards a complex networks of quantum technologies, the capacity to reliably certify quantum technologies comes as a necessity.
Certification of resources would enable quick detection of errors and provide guarantees on the behaviour of quantum devices, crucial for security-sensitive applications.
Interestingly, certification protocols can be formulated in a \textit{device-independent} manner, i.e. without assumption on the behaviour of quantum devices. 
\textit{Self-testing} is a particularly noteworthy protocol that provides a guarantee on the presence of specific states and measurements.
Alternatively, for certain quantum applications, protocols can be devised that only succeed if resources requirements are met.
\textit{Device-independent quantum key distribution} is of prime example of this, allowing two parties to share a secret key, with a security proof that does not rely on assumptions on the quantum devices.
In the presence of interferences, whether from adversarial attacks or from faulty quantum devices, DIQKD protocols will simply abort.

\medbreak

In this thesis, we focus on these device-independent protocols, with an emphasis on enabling their experimental implementations.
Indeed, while these protocols are promising, their realization requires meeting high requirements.
Therefore, we begin by quantifying the resources necessary for their successful implementation.
We then improve the robustness of these protocols, ultimately lowering the requirements for their implementations.
Finally, we propose a method to generate new experimental designs which led to proposals well-suited for their successful realization.

\section{Summary of the contributions}

\paragraph{Device-independent certification of quantum resources} 

Self-testing enables the certification of quantum resources from observed measurement statistics.
We focus on robust singlet self-testing, the self-test of two-qubit maximally entangled states in the presence of losses and noises.

\medbreak

\textbf{Article 1~\cite{Valcarce2020}:} We quantify resources that are necessary for robust self-tests protocols to apply.
We focus on self-test based on a single condition on the measurement statistics; the violation of the CHSH inequality.
Our work lead to the discovery of a no-go condition, a minimal violation below which this robust self-test fails.
This fundamental limit improves our understanding of self-testing, shows clear experimental requirements and suggests directions for future robust singlet self-testing protocols.

\textbf{Article 2~\cite{Valcarce2022}:} From a refined analysis of the measurement statistics, we derive new robust singlet self-tests.
The new protocol we propose provides a better robustness to losses and thus eases experimental realization of singlet self-testing.

\paragraph{Device-independent quantum key distribution}

Device-independent quantum key distribution protocols allow two parties to share a key in a provably secure way.
Their security proof rely on conditions on the observed measurement statistics. 

\medbreak

\textbf{Article 3~\cite{Sekatski2021}:} We introduce a new security proof, bounding more tightly the information an eavesdropper could gather on the key.
Our proof is based on two correlations functions instead of a linear combination of all correlators.
The resulting key rate is higher for realistic photonic implementations and partially entangled states but does not provide a better critical efficiency.

\textbf{Article 4~\cite{Valcarce2022b}:} We automize the design of photonic experiments by combining machine learning and a custom-made simulation framework~\cite{Valcarce2021}. Applied to DIQKD, our method results in new realistic experimental blue prints, yielding both a higher key rate and a higher tolerance to losses compared to known photonic proposals.

