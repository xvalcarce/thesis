\part{Overview and definitions}

%In this chapter, we review some key concepts that are used throughout this thesis. 
%At the core of quantum information, the notions of quantum entanglement and non-locality are introduced in Sections \ref{section:entanglement} and \ref{section:nonlocality} respectively. 
%Self-testing
%Device-independent certification

\chapter{Quantum Entanglement}
\label{section:entanglement}

Quantum entanglement is often thought has one of the most bizarre feature of quantum mechanics. 
In 1935, it first puzzled \acrfull{epr} who described the phenomena in \cite{Einstein35}. 
The same year, Schrödinger, in the seminal paper \cite{Schrödinger35},
coined the term \textit{entanglement} and portrayed it as not \enquote{\textit{one} but rather \textit{the} characteristic trait of quantum mechanics, the one that enforces its entire departure from classical lines of thought.}

An entangled system is described in opposition to a separable system, i.e. a system that can be fully described from the state of its individual components. 
Such entangled system are allowed by the \textit{superposition} principle and the structure of the space of joint quantum systems. 

A pure quantum state $\ket{\psi}$ is represented by a vector in a Hilbert space $\mathscr{H}$.
Given two pure state $\ket{\psi^A}$ and $\ket{\psi^B}$ and their respective Hilbert space $\mathscr{H}^A$ and $\mathscr{H}^B$, a separable system composed of these state can be written as the tensor product of its components
\begin{equation}
	\ket{\psi} = \ket{\psi^A} \otimes \ket{\psi^B}
	\label{eq:separable_state}
\end{equation}
associated with the Hilbert space $\mathscr{H}=\mathscr{H}^A\otimes\mathscr{H}^B$. 
Conversely a state in $\mathscr{H}^{AB}$ that can not be written in the previous form is said to be entangled.
The four Bell states 
\begin{align}
	\ket{\psi^+} &= \frac{1}{\sqrt{2}}\left(\ket{01}+\ket{10}\right) \\
	\ket{\psi^-} &= \frac{1}{\sqrt{2}}\left(\ket{01}-\ket{10}\right) \\
	\ket{\phi^+} &= \frac{1}{\sqrt{2}}\left(\ket{00}+\ket{11}\right) \\
	\ket{\phi^-} &= \frac{1}{\sqrt{2}}\left(\ket{00}-\ket{11}\right)
	\label{eq:bell_states}
\end{align}
are famous example of such entangled states.

More generally, a convex sum of pure states, or \textit{mixed state}, described by the density matrix
\begin{equation}
	\rho_{AB} = \sum_i c_i \ket{\psi_i} \bra{\psi_i} \qquad \text{with} \quad \sum_i c_i = 1
	\label{eq:mixed_state}
\end{equation}
is said to be entangled if it can not be decomposed as a convex sum of product states
\begin{equation}
	\rho_{AB} = \sum_i c_i \rho^A_i \otimes \rho^B_j.
	\label{eq:product_state}
\end{equation}

\medbreak
The phenomena of entanglement leads to counter intuitive predictions. 
Most notably, the fact that a property measured on one part of an entangled system can determine the measurement outcomes on the other part of such system, even if both parties are far apart. 
This was famously refer to as \enquote{spooky action at distance} by Einstein and leads to a paradox, the \acrshort{epr} paradox.

Quantum mechanics imposes that the values of two non-commuting observables can not be known simultaneously. 
For example, it is impossible to both exactly know the spin value in the $x$-direction and in the $z$-direction of system, since these two spin observables do not commute.
The \acrshort{epr} paradox occurs when measuring non-commuting observables on an entangled system.
Consider the entangled state $\ket{\psi^+}$ shared between Alice and Bob.
If Alice where to measure the spin in the $z$-direction of a $\ket{\psi^+}$ state and obtain the value $s_z = 1/2$, she can assume that if Bob where to measure his system in the same manner he would obtain $s_z = 1/2$. 
So it seems that if Bob measures his system in the $x$-direction and ask Alice for the outcome of her measurement in the $z$-direction, he would no precisely the values of two non-commuting observables.

Believing in the complete local determination of the outcomes, \acrlong{epr} proposed the existence of \textit{hidden} variables to solve the paradox. Another solution to the paradox is that physics is not compatible with \textit{local realism}.

\medbreak
Entanglement was at first left by physicists as a purely fundamental and philosophical problem of quantum mechanics.
In 1964, John Bell proved that quantum mechanics can not be explained by a physiscal theory of \textit{local hidden variable}, under the assumption of \textit{free will}. 
Based on the work of John Bell, Clauser, Horne, Shimony and Holt (\acrshort{chsh}) porposed an experimental protocol to test Bell's proof. 
This proposal was soon followed by a first experimental realisation showing evidence towards Bell's proof.
In the early 80's, Alain Aspect and his colleagues performed a series of experiments, ensuring that the assumption of \textit{free will} is reasonnable.
Finally, in 1998, an experiment performed in the group of Anton Zeilinger, closed the last loophole by ensuring local conditions.
These works were utlimately recognized by the 2022 Nobel prize, awarded to John Clauser, Alain Aspect and Aton Zeilinger.

\medbreak
The work of Bell followed by experimental evidences brought back interest on entanglement.
In particular, the capabality to generate entangled pair of particles led into thinking of entanglement as a quantum ressource.
This allowed for a variety of new technologies and the creation of a subfiled of physics known today as \textit{quantum information}.
Entanglement-based technologies include quantum key distribution, quantum computing, quantum random numbers generators, and much more. 


\chapter{Non-locality}
\label{section:nonlocality}


\section{Bell game}

In order to answer the \acrshort{epr} paradox, John Bell introduced a theoretical game, known today as a \textit{Bell game}.
Such a game consists of two players, Alice and Bob, that may have been in contact in the past but that are spatially separated while the game is played.
A spatial separation ensures that their is no communication between the two parties while the game is played.
Alice and Bob each own a physical system as well as a finite number of measurement devices.

Each round of the game, Alice chooses a measurement to perform and records the outcome of that measurement on her system.
We label $x$ her \textit{measurement choice} or \textit{setting}, $A_x$ the corresponding measurement, and $a$ the outcome.
Similarly Bob makes a measurement choice $y$ and obtains the outcome $b$ from the measurement $B_y$.
This Bell game is depicted in Fig..

The outcome $a,b$ are not necessarily deterministic function of the inputs $x,y$, we are thus interested in the  probability $p(ab|xy)$. 
For a given game, one can arrange the probabilities for every possible outcomes and inputs in a vector $\mathbf{P}=\{ p(ab|xy)\}$ also refer to as \textit{correlations}.
These correlations form a set whose boundaries depends on which assumptions are fulfilled by the Bell game and which resources are considered.


\section{Non-signalling correlations}

\textit{Non-signalling} correlations are correlations respecting a single assumption: the measurement choice of one party can not influence the outcome of the other party's measurement.
When Alice and Bob are space-like separated, this assumption is enforced by special relativity, i.e. no information can be transmitted faster-than-light.

Formally, non-signalling correlations are correlations for which the local marginals of a party are independent of the other party's measurements choice. 
Mathematically, such correlations satisfy
\begin{align}
	\sum_b p(ab|xy) = \sum_b p(ab|xy') = p(a|x), \quad &\forall\,a,x,y,y' \\
	\sum_a p(ab|xy) = \sum_a p(ab|x'y) = p(b|y), \quad &\forall\,b,x,x',y.
	\label{eq:non-signalling}
\end{align}


\section{Local and non-local correlations}

\textit{Local correlations} are fully characterised from the local system of Alice and Bob.
This type of correlation accounts for some hypothetical shared hidden information between the two parties.
Indeed, since Alice and Bob were in contact in the past, outcomes of their measurements can be influenced by some past factors, available locally to both parties' systems but possibly unobservable.
These past factors are \textit{local-hidden variables} with arbitrary value that we represent using $\lambda$.
Moreover, because these variables might not stay constant throughout the game, we denote $q(\lambda)$ the probability distribution governing their evolution. 
From here, we can formally defined local correlations as any correlation that can be decomposed as
\begin{equation}
	p(ab|xy) = \int_\lambda \mathrm{d}\lambda \, q(\lambda) p(a|x,\lambda) p(b|y,\lambda).
	\label{eq:local}
\end{equation}
One assumption we made in writing this decomposition is the \textit{freedom of choice} of both parties' measurements, as $x$ and $y$ are independent of $\lambda$.

Interestingly, any local correlation can be written as a convex sum of deterministic local strategies, $p_i(ab|xy)=p(a|x,i)p(b|y,i)$, following
\begin{equation}
	p(ab|xy) = \sum_i c_i p_i(ab|xy)
	\label{eq:polytope}
\end{equation}
where $c_i\geq 0$ and $\sum_i c_i = 1$.
From the previous decomposition it follows that the convex sum of local strategies is also local. 
Geometrically, the set of local correlations is the convex hull of deterministic strategies which is a convex polytope, the local polytope.

\medbreak

\textit{Non-local} correlations are defined as any correlation that can not admit be written in the form of \refeq{local}.
Such correlations can thus not be explained by means of local-hidden variable. 
Note that this is the case for some non-signalling correlations as correlations satisfying \refeq{non-signalling} do not necessarily admit a decomposition of the form \refeq{local}.
Conversely, local correlations respect the non-signalling assumption.

\section{Quantum correlations}

\textit{Quantum correlations} occur when Alice and Bob each share a part of a quantum system, $\rho_{AB}$, that they prepared when they were in contact.
Generally, this quantum state is a mixed state laying in the Hilbert space $\mathscr{H}^{AB} = \mathscr{H}^{A} \otimes \mathscr{H}^B$.
For measurements choice $x,y$, quantum correlations are given by Born's rule
\begin{equation}
	p(ab|xy) = \trr{\rho_{AB} M^a_x \otimes M^b_y} \quad \forall\,a,b,x,y
	\label{eq:Born}
\end{equation}
where $M^a_x$ is the \acrfull{povm} element of measurement $A_x$ associated with outcome $a$, and similarly for Bob with the \acrshort{povm} $M^b_y$ for measurement $B_y$ and outcome $b$.

Quantum correlations satisfy the non-signalling assumption, but are not necessarily local.
In order to obtain non-local quantum correlations, Alice and Bob must share an entangled state.
Entanglement can thus be detected from the presence of a non-local correlation, given that the non-signalling assumption is enforced.

\section{Bell inequalities}

\section{Non-locality as a quantum ressource}

\subsection{Self-testing}
\subsection{Device-independent certification}

